\documentclass[a4paper, 11pt]{article}
\usepackage{amsmath}
\usepackage{graphicx}
\usepackage{geometry}
\usepackage{listings}
\usepackage{xcolor}
\usepackage[colorlinks,linkcolor=red]{hyperref}
\geometry{scale=0.8}
\usepackage[UTF8]{ctex}
\title{	
\normalfont \normalsize
\textsc{School of Data and Computer Science, Sun Yat-sen University} \\ [25pt] %textsc small capital letters
\rule{\textwidth}{0.5pt} \\[0.4cm] % Thin top horizontal rule
\huge  E09 Variable Elimination \\ % The assignment title
\rule{\textwidth}{2pt} \\[0.5cm] % Thick bottom horizontal rule
\author{18340052  何泽}
\date{\normalsize\today}
}

\begin{document}
\maketitle
\tableofcontents
\newpage


\section{VE}

The burglary example is described as following:
\begin{figure}[h]
  \centering

  \includegraphics[width=14cm]{Pic/burglary}
\end{figure}

\begin{figure}[ht]
\centering
\includegraphics[width=12cm]{Pic/burglar_result}
\end{figure}
Here is a VE template for you to solve the burglary example:
\begin{lstlisting}[language=Python,frame=single]
class VariableElimination:
    @staticmethod
    def inference(factorList, queryVariables, 
    orderedListOfHiddenVariables, evidenceList):
        for ev in evidenceList:
            #Your code here
        for var in orderedListOfHiddenVariables:
            #Your code here
        print "RESULT:"
        res = factorList[0]
        for factor in factorList[1:]:
            res = res.multiply(factor)
        total = sum(res.cpt.values())
        res.cpt = {k: v/total for k, v in res.cpt.items()}
        res.printInf()
    @staticmethod
    def printFactors(factorList):
        for factor in factorList:
            factor.printInf()
class Util:
    @staticmethod
    def to_binary(num, len):
        return format(num, '0' + str(len) + 'b')
class Node:
    def __init__(self, name, var_list):
        self.name = name
        self.varList = var_list
        self.cpt = {}
    def setCpt(self, cpt):
        self.cpt = cpt
    def printInf(self):
        print "Name = " + self.name
        print " vars " + str(self.varList)
        for key in self.cpt:
            print "   key: " + key + " val : " + str(self.cpt[key])
        print ""
    def multiply(self, factor):
        """function that multiplies with another factor"""
        #Your code here
        new_node = Node("f" + str(newList), newList)
        new_node.setCpt(new_cpt)
        return new_node
    def sumout(self, variable):
        """function that sums out a variable given a factor"""
        #Your code here
        new_node = Node("f" + str(new_var_list), new_var_list)
        new_node.setCpt(new_cpt)
        return new_node
    def restrict(self, variable, value):
        """function that restricts a variable to some value 
        in a given factor"""
        #Your code here
        new_node = Node("f" + str(new_var_list), new_var_list)
        new_node.setCpt(new_cpt)
        return new_node
# create nodes for Bayes Net
B = Node("B", ["B"])
E = Node("E", ["E"])
A = Node("A", ["A", "B","E"])
J = Node("J", ["J", "A"])
M = Node("M", ["M", "A"])

# Generate cpt for each node
B.setCpt({'0': 0.999, '1': 0.001})
E.setCpt({'0': 0.998, '1': 0.002})
A.setCpt({'111': 0.95, '011': 0.05, '110':0.94,'010':0.06,
'101':0.29,'001':0.71,'100':0.001,'000':0.999})
J.setCpt({'11': 0.9, '01': 0.1, '10': 0.05, '00': 0.95})
M.setCpt({'11': 0.7, '01': 0.3, '10': 0.01, '00': 0.99})

print "P(A) **********************"
VariableElimination.inference([B,E,A,J,M], ['A'], ['B', 'E', 'J','M'], {})

print "P(B | J~M) **********************"
VariableElimination.inference([B,E,A,J,M], ['B'], ['E','A'], {'J':1,'M':0})
\end{lstlisting}
\section{Task}
\begin{itemize}
\item You should implement 4 functions: \texttt{inference}, \texttt{multiply}, \texttt{sumout} and \texttt{restrict}. You can turn to Figure \ref{Fig:ve_product} and Figure \ref{Fig:sumout_restrict} for help. 
\item Please hand in a file named \textsf{E09\_YourNumber.pdf}, and send it to \textsf{ai\_2020@foxmail.com}
\end{itemize}


\begin{figure}[ht]{}
\centering
\includegraphics[width=7.5cm]{Pic/ve}
\qquad
\includegraphics[width=7.5cm]{Pic/product}
\caption{VE and Product}
\label{Fig:ve_product}
\end{figure}


\begin{figure}[htb]
\centering
\includegraphics[width=7.5cm]{Pic/sumout}
\qquad
\includegraphics[width=7.5cm]{Pic/restrict}
\caption{Sumout and Restrict}
\label{Fig:sumout_restrict}
\end{figure}



\section{Codes and Results}
Code:
\lstset{
	columns=fixed,       
	numbers=left,                                        % 在左侧显示行号
	frame=shadowbox,                                          % 不显示背景边框
	backgroundcolor=\color[RGB]{245,245,244},            % 设定背景颜色
	keywordstyle=\color[RGB]{0,92,230},                 % 设定关键字颜色
	numberstyle=\footnotesize\color{darkgray},           % 设定行号格式
	commentstyle=\it\color[RGB]{0,96,96},                % 设置代码注释的格式
	stringstyle=\rmfamily\slshape\color[RGB]{230,92,0},   % 设置字符串格式
	showstringspaces=false,                              % 不显示字符串中的空格
	breaklines,
	language=Python,      
}
\begin{lstlisting}
class VariableElimination:
    @staticmethod
    def inference(factorList, queryVariables,
                  orderdListOfHiddenVariables, evidenceList):
        for ev in evidenceList:
            for factor in factorList:
                if ev in factor.varList:
                    factorList.append(factor.restrict(ev, evidenceList[ev]))
                    factorList.remove(factor)

        for var in orderdListOfHiddenVariables:
            eliminationList = list(filter(lambda factor: var in factor.varList,factorList))
            new_var = eliminationList[0]
            for e in eliminationList:
                for i in factorList:
                    if i.name == e.name:
                        factorList.remove(i)
                if not e == eliminationList[0]:
                    new_var = new_var.multiply(e)
            new_var = new_var.sumout(var)
            factorList.append(new_var)
        print("RESULT:")
        res = factorList[0]
        for factor in factorList[1:]:
            res = res.multiply(factor)
        total = sum(res.cpt.values())
        res.cpt = {k: v/total for k, v in res.cpt.items()}
        res.printInf()

    @staticmethod
    def printFactors(factorList):
        for factor in factorList:
            factor.printInf()


class Util:
    @staticmethod
    def to_binary(num, len):
        return format(num, '0' + str(len) + 'b')


class Node:
    def __init__(self, name, var_list):
        self.name = name
        self.varList = var_list
        self.cpt = {}

    def setCpt(self, cpt):
        self.cpt = cpt

    def printInf(self):
        print("Name = " + self.name)
        print(" vars " + str(self.varList))
        for key in self.cpt:
            print("   key: " + key + " val : " + str(self.cpt[key]))
        print()

    def multiply(self, factor):
        newList = [var for var in self.varList]
        new_cpt = {}
        
        idx1 = []
        idx2 = []
        for var1 in self.varList:
            for var2 in factor.varList:
                if var1 == var2:
                    idx1.append(self.varList.index(var1))
                    idx2.append(factor.varList.index(var2))
                else:
                    newList.append(var2)

        for k1, v1 in self.cpt.items():
            for k2, v2 in factor.cpt.items():
                flag = True
                for i in range(len(idx1)):
                    if k1[idx1[i]] != k2[idx2[i]]:
                        flag = False
                        break
                if flag:
                    new_key = k1
                    for i in range(len(k2)):
                        if i in idx2: continue
                        new_key += k2[i]
                    new_cpt[new_key] = v1 * v2
        new_node = Node("f" + str(newList), newList)
        new_node.setCpt(new_cpt)
        return new_node

    def sumout(self, variable):
        new_var_list = [var for var in self.varList]
        new_var_list.remove(variable)
        new_cpt = {}
        idx = self.varList.index(variable)
        for k, v in self.cpt.items():
            if k[:idx] + k[idx+1:] not in new_cpt.keys():
                new_cpt[k[:idx] + k[idx+1:]] = v
            else: new_cpt[k[:idx] + k[idx+1:]] += v
        new_node = Node("f" + str(new_var_list), new_var_list)
        new_node.setCpt(new_cpt)
        return new_node

    def restrict(self, variable, value):
        new_var_list = [i for i in self.varList]
        new_var_list.remove(variable)
        new_cpt = {}
        idx = self.varList.index(variable)
        for k, v in self.cpt.items():
            if k[idx] == str(value):
                new_cpt[k[:idx] + k[idx+1:]] = v
        new_node = Node("f" + str(new_var_list), new_var_list)
        new_node.setCpt(new_cpt)
        return new_node

B = Node("B", ["B"])
E = Node("E", ["E"])
A = Node("A", ["A", "B", "E"])
J = Node("J", ["J", "A"])
M = Node("M", ["M", "A"])

B.setCpt({'0': 0.999, '1': 0.001})
E.setCpt({'0': 0.998, '1': 0.002})
A.setCpt({'111': 0.95, '011': 0.05,'110': 0.94, '010': 0.06,
            '101': 0.29, '001': 0.71,'100': 0.001, '000': 0.999})
J.setCpt({'11': 0.9, '01': 0.1,'10': 0.05, '00': 0.95})
M.setCpt({'11': 0.7, '01': 0.3,'10': 0.01, '00': 0.99})

print("P(A) **********************")
VariableElimination.inference(
        [B, E, A, J, M], ['A'],
        ['B', 'E', 'J', 'M'],{})

print("P(J && ~M) **********************")
VariableElimination.inference(
        [B, E, A, J, M], ['J'],
        ['B', 'E', 'A'],{'M': 0})

print("P(A | J && ~M) **********************")
VariableElimination.inference(
        [B, E, A, J, M], ['A'],
        ['B', 'E'],{'J': 1, 'M': 0})

print("P(B | A) **********************")
VariableElimination.inference(
        [B, E, A, J, M], ['B'],
        ['E', 'J', 'M'],{'A': 1})

print("P(B | J && ~M) **********************")
VariableElimination.inference(
        [B, E, A, J, M], ['B'],
        ['E', 'A'],{'J': 1, 'M': 0})

print("P(J && ~M | ~B) **********************")
VariableElimination.inference(
        [B, E, A, J, M], ['J', 'M'],
        ['E', 'A'],{'B': 0})
\end{lstlisting}
Result:
\begin{figure}[ht]
  \centering
  \includegraphics[width=0.9\textwidth]{Pic/1.png}
  \qquad
\end{figure}


%\clearpage
%\bibliography{E:/Papers/LiuLab}
%\bibliographystyle{apalike}
\end{document} 
%%% Local Variables:
%%% mode: latex
%%% TeX-master: t
%%% End:
